%\documentclass[aps,prl,preprint,groupedaddress]{revtex4}
%\documentclass[aps,prl,preprint,superscriptaddress]{revtex4}
\documentclass[aps,pra,preprint,groupedaddress]{revtex4}
\usepackage{graphicx}
\usepackage{array}
\usepackage{amsmath}
\usepackage{verbatim}

\begin{document}

%Title of paper
\title{Inner Time and The Inner ear}

% lots of work on CPT before considering EIT
%CPT--> STIRAP, Sub-recoil cooling
%mention paper of observation of CP1T?
%lasing without inversion, more Harris stuff pre-91?
\author{Timur Rvachov}
%\affiliation{Department of Physics, Massachusetts Institute of Technology, Cambdirge, MA}
%\affiliation{Department of Physics, Massachusetts Institute of Technology, Cambdirge, MA, USA,
%Earth, Milky Way Galaxy, Local Group, Virgo Supercluster, Observable Universe (relative to Earth) }
%
\date{October 15, 2012}

\begin{abstract}

This document will contain the relevant equations I need to numerically calculate the Paschen-Back for aribitrary $I$ and $J$.
\end{abstract}

\maketitle

%New command definitions
\newcommand{\IdotJ}{\mathbf{I}\cdot \mathbf{J}}

% Body
\section {The whole paper}\label{intro}
The hyperfine splitting up to magnetic octopole contributions is given by:
\begin{equation}
\begin{array}{l}
H_\text{hfs}=A_\text{hfs}\dfrac{\IdotJ}{\hbar^2} \\
+B_\text{hfs}\dfrac{\dfrac{3}{\hbar^2}(\IdotJ)^2+\dfrac{3}{2\hbar}\mathbf{I}\cdot \mathbf{J}-I(I+1)J(J+1)}{2I(2I-1)J(2J-1)}\\
+C_\text{hfs}\dfrac{\dfrac{10}{\hbar^3}(\IdotJ)^3+\dfrac{20}{\hbar^2}(\IdotJ)^2+\dfrac{2}{\hbar}\IdotJ\left[I(I+1)+J(J+1)+3\right]-3I(I+1)J(J+1)-5I(I+1)J(J+1)}{I(I-1)(2I-1)J(J-1)(2J-1)}
\end{array}
\end{equation}
For Sodium 23, we have:

\begin{equation}
\begin{array}{lll}
\text{Magnetic Dipole Constant, $3^2S_{1/2}$}&A_{3^2S_{1/2}}&h\cdot 885.813\text{ MHz}\\
\text{Magnetic Dipole Constant, $3^2P_{1/2}$}&A_{3^2P_{1/2}}&h\cdot 94.44\text{ MHz} \\
\text{Magnetic Dipole Constant, $3^2P_{3/2}$}&A_{3^2P_{3/2}}&h\cdot 18.534\text{ MHz}\\
\text{Electric Quadrupole Constant, $3^2P_{3/2}$}&B_{3^2P_{3/2}}&h\cdot 2.724\text{ MHz}\\
\end{array}
\end{equation}

For the DC Zeeman shift, we want to compute the eigenstaes of
\begin{equation}
H_B^\text{(fs)}+H_B^\text{(hfs)}
\end{equation}
where for a $B$-field in the $z$ direction, we have
\begin{equation}
H_B^\text{(fs)}=-\mathbf{\mu}_S\cdot \mathbf{B} - \mathbf{\mu}_L\cdot \mathbf{B}
=\dfrac{\mu_B}{\hbar}\left(g_s S_z+g_L L_z\right)B
\end{equation}
and
\begin{equation}
H_B^\text{(hfs)}=-\mathbf{\mu}_I\cdot \mathbf{B}=\dfrac{\mu_B}{\hbar}g_I I_z B.
\end{equation}

For now, we will lump together the effect of $S$ and $L$ into $J$ via the Lande $g_J$ factor, and treat only the Paschen-Back effect due to the hyperfine interaction (since this occurs at much lower energy $B$-fields than the separation of $S$ and $L$ eigenstates). Thus we must diagonalize
\begin{equation}
H_B^\text{(fs)}+H_B^\text{(hfs)}=\dfrac{\mu_B}{\hbar}\left( g_J J_z+g_I I_z\right)B
\end{equation}
where
\begin{equation}
\begin{array}{l}
g_J \equiv g_L + (g_S-g_L)\dfrac{J(J+1)+S(S+1)-L(L+1)}{2J(J+1)}\\
g_L \equiv \dfrac{\text{Reduced Mass}}{m_e}=\dfrac{1}{1+m_e/m_n} \\
g_S=2.002319.
\end{array}
\end{equation}

In order to compute matrix elements of $\IdotJ$ in $H_B^\text{(hfs)}$, we will need to use:
\begin{equation}
\begin{array}{l}
\IdotJ=I_z J_z+\dfrac{I_+J_-+I_-J_+}{2}\\
(\IdotJ)^2=(I_zJ_z)^2+\dfrac{1}{2}\{I_zJ_z,I_+J_-+I_-J_+\}+\dfrac{(I_+J_-)^2+(I_-J_+)^2}{4}+\dfrac{I_+I_-J_-J_++I_-I_+J_+J_-}{4}
\end{array}
\end{equation}



\bibliography{amo_paper}

\end{document}

